\documentclass{beamer}

\usepackage[utf8]{inputenc}
\usepackage{graphicx}
\usepackage{xcolor,colortbl}
\usepackage{subfigure}
%\usepackage{columns}
\usepackage{default}
\usetheme{Frankfurt}
\usecolortheme{seahorse}

\setbeamertemplate{footline}[page number]


%\usecolortheme{whale}
\centerline{\includegraphics[scale=0.4]{../images/easy_elua_logo}}

\title{Projet innovant RICM4: Easy-eLua}
\author{Elizabeth \textsc{Paz} \\ Salem \textsc{Harrache}}
\institute{Polytech'Grenoble \\
Olivier \textsc{Richard} \\
Didier \textsc{Donsez} \\
}

\author[Elizabeth Paz, Salem Harrache]
{Elizabeth Paz \and Salem Harrache}

\pgfdeclareimage[height=1cm]{university-logo}{../images/polytech.png}
\logo{\pgfuseimage{university-logo}}
\date{27 Avril 2012}

\begin{document}


\begin{frame}
 \maketitle
\end{frame}

\begin{frame}
\frametitle{Sommaire}
\tableofcontents
\end{frame}

\section{Introduction}
\subsection{Présentation carte STM32F4-DISCOVERY}
\subsection{Présentation Arduino}
\subsection{Présentation eLua}
\section{Travail réalisé}
\subsection{Organisation du projet}
\subsection{Fonctions porté}
\subsection{Nouveaux concepts}
\section{Demonstration}
\section{Hello Word!}
\section{Blink with button}
\section{Conclusion}
\subsection{Difficulté}


\end{document}