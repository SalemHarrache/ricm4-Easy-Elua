\documentclass{beamer}

\usepackage[utf8]{inputenc}
\usepackage{graphicx}
\usepackage{xcolor,colortbl}
\usepackage{subfigure}
\usepackage{hyperref}
%\usepackage{columns}
\usepackage{default}
\usetheme{Frankfurt}
\usecolortheme{seahorse}
\usepackage{listings}
\lstset{
    tabsize=4,
    rulecolor=,
    language=Python,
    basicstyle=\scriptsize,
    upquote=true,
    aboveskip={1.5\baselineskip},
    columns=fixed,
    showstringspaces=false,
    extendedchars=true,
    breaklines=true,
    prebreak = \raisebox{0ex}[0ex][0ex]{\ensuremath{\hookleftarrow}},
    frame=single,
    showtabs=false,
    showspaces=false,
    showstringspaces=false,
    identifierstyle=\ttfamily,
    keywordstyle=\color[rgb]{0,0,1},
    commentstyle=\color[rgb]{0.133,0.545,0.133},
    stringstyle=\color[rgb]{0.627,0.126,0.941},
}

\setbeamertemplate{footline}[page number]


%\usecolortheme{whale}
%\centerline{\includegraphics[scale=0.4]{../images/easy_elua_logo}}


\title{Projet innovant RICM4: Easy-eLua}
\author{Elizabeth \textsc{Paz} \\ Salem \textsc{Harrache}}
\institute{Polytech'Grenoble \\
Olivier \textsc{Richard} \\
Didier \textsc{Donsez} \\
}

\author[Elizabeth Paz, Salem Harrache]
{Elizabeth Paz \and Salem Harrache}

\pgfdeclareimage[height=1cm]{university-logo}{../images/polytech.png}
\logo{\pgfuseimage{university-logo}}
\date{27 Avril 2012}

\begin{document}

\begin{frame}
\begin{center}
\includegraphics[scale=0.4]{../images/easy_elua_logo}
\end{center}
\titlepage
\end{frame}

\begin{frame}
\frametitle{Sommaire}
\tableofcontents
\end{frame}

\section{Introduction}
%\subsection{Présentation carte STM32F4-DISCOVERY}
%\begin{frame}
%\frametitle{Introduction : Présentation carte STM32F4-DISCOVERY}
%Partie Elizabeth
%Note salem : présenter brièvement les caractéristiques et puis finir sur le fait que
%c'est compliqué de programmer dessus pour un novice
%\begin{center}
 %\includegraphics[scale=0.1]{../images/stm32f4_discovery.jpg}
%\end{center}
%\end{frame}

\subsection{Présentation Arduino}
\begin{frame}
\frametitle{Introduction : Présentation Arduino}
\textbf{Système Arduino: } plateforme open-source de programmation embarquée, basée sur une carte
à microcontrôleur de la famille AVR. \\
\vspace{0.5cm}
\textbf{Arduino permet de réaliser: } du prototypage rapide, la domotique, communication avec des logiciels, etc \ldots \\
\vspace{0.5cm}
\textbf{Principales avantages: } peu coûteux, multi-plateforme, le logiciel et le matériel sont open source et extensibles, etc \ldots \\
\vspace{0.5cm}
\textbf{Utilisation: } deux méthodes à implementer \textit{obligatoirement}: loop() et setup().

\end{frame}

\subsection{Présentation eLua}
\begin{frame}
\frametitle{Introduction : Présentation eLua}

\textbf{Lua} est un langage de script libre, réflexif et impératif. But: pouvoir être embarqué au sein d'applications et les étendre. \\
\vspace{0.5cm}
\textbf{eLua:} adopte le langage de programmation Lua et propose une implémentation complète de celui-ci pour les systèmes embarqués. \\

\begin{center}
 \includegraphics[scale=0.55]{../images/eLua/Lua.JPG} \ \quad
 \includegraphics[scale=0.3]{../images/eLua/logo_eLua.png}
\end{center}
\end{frame}

\begin{frame}
\frametitle{Introduction : Présentation eLua}

\textbf{Principales avantages de l'utilisation de eLua: } un contrôle absolu des plateformes, portabilité du code, développement autonome, flexibilité,
open source, etc \ldots \\
\vspace{0.5cm}
\textbf{Architecture de eLua:} composé de façon a être le plus portable possible en suivant de près certaines de règles et ayant toujours du code qui sera
commun a toutes les plateformes.

\end{frame}


%Schema
\begin{frame}
\frametitle{Introduction : Présentation eLua}
\begin{center}
 \includegraphics[scale=0.55]{../images/eLua/schema.png}
\end{center}
\end{frame}

\section{Travail réalisé}
\subsection{Organisation du travail}
\begin{frame}
\frametitle{Organisation du travail : Agilité et Autodidacte}
\begin{enumerate}
 \item Phase I
\begin{itemize}
\item Premiers sprints assez longs et peu productifs
\begin{itemize}
\item Recherche d'informations sur eLua et d'éventuels portages pour STM32
\item Evaluation de la faisabilité
\item Initiation à la programmation avec la lib stlink sur linux
\end{itemize}
\item Concertation hebdomadaire sur l'avancement
\item Discussions avec notre tuteur de stage
\end{itemize}
 \item Phase II
\begin{itemize}
\item Sprints très courts
\item Conception de l'architecure du projet
\item Développement d'outils de travail (installation, flash)
\item Utilisation d'un gestionnaire de version
\end{itemize}
\end{enumerate}
\end{frame}

\subsection{Arboresence du projet}
\begin{frame}
\frametitle{Arboresence du projet}
\begin{center}
 \includegraphics[scale=0.5]{../images/root_dir_project.png}
\end{center}
\end{frame}

\subsection{Fonctions portées}
\begin{frame}
\frametitle{Fonctions portées (1)}
\begin{enumerate}
 \item Entrées/Sorties numériques
\begin{itemize}
\item pinMode() $\to$ déclarer les broches en entrée ou en sortie
\item digitalWrite() $\to$ écriture d'une valeur HIGH/LOW (1/0)
\item digitalRead() $\to$ lecture
\end{itemize}
\item Communication Série
\begin{itemize}
\item Serial::begin() $\to$ initialiser la connexion série
\item Serial::read()
\item Serial::write()
\item Serial::print()
\end{itemize}
\item Time
\begin{itemize}
\item millis()  $\to$ Durée d'exécution du programme
\item micros()
\item delay()  $\to$ Attente passive
\item delayMicroseconds()
\end{itemize}
\end{enumerate}
\end{frame}

\begin{frame}
\frametitle{Fonctions portées (2)}
\begin{table}[h]
 \begin{tabular}{|l|l|} \hline
\textbf{Arduino} & \textbf{Easy-eLua} \\ \hline
Serial::begin() $\to$ & SerialPort:begin() \\ \hline
Serial::available() $\to$ & SerialPort:readWait() \\
                          & SerialPort:read() \\ \hline
Serial::read() $\to$ & SerialPort:read() \\ \hline
Serial::write() $\to$ & SerialPort:write() \\ \hline
Serial::print() $\to$ & SerialPort:print() \\ \hline
Serial::println() $\to$ & SerialPort:println() \\ \hline
Serial::end() $\to$ & \\ \hline
Serial::flush() $\to$ & \\ \hline
 \end{tabular}
\end{table}
\end{frame}

\subsection{Nouveaux concepts}
\begin{frame}[containsverbatim]
\frametitle{Nouveaux concepts}
\begin{enumerate}
 \item Programmation orienteé objects
 \item Lua et la métaprogrammation $\to$ Redéfinition du type ``Class``
 \item Introduction de l'objet App qui s'exécute avec un contexte
\end{enumerate}

\scriptsize{\begin{lstlisting}
App = Class:new()
function App:setup()
    -- The setup function will only run once after each
    -- powerup or reset of the board
end
function App:loop()
    -- loops consecutively
end

function App:run()
    self:setup()
    while condition do
        self:loop()
    end
end
\end{lstlisting}}

\end{frame}

\begin{frame}[containsverbatim]
\frametitle{Exemple : ''Blink`` (Lua)}
\tiny{\begin{lstlisting}
require("arduino_wrapper")

function App:setup()
    self.ledpin = ORANGE_LED
    pinMode(self.ledpin, OUTPUT)
end

function App:loop()
    digitalWrite(self.ledpin, HIGH)
    delay(1000)
    digitalWrite(self.ledpin, LOW)
    delay(1000)
end

app = App:new("Blink led")
app:run()
\end{lstlisting}}
\end{frame}

\begin{frame}[containsverbatim]
\frametitle{Exemple : ''Blink`` (Arduino)}
\tiny{\begin{lstlisting}

void setup() {
  // initialize the digital pin as an output.
  // Pin 13 has an LED connected on most Arduino boards:
  pinMode(13, OUTPUT);
}

void loop() {
  digitalWrite(13, HIGH);   // set the LED on
  delay(1000);              // wait for a second
  digitalWrite(13, LOW);    // set the LED off
  delay(1000);              // wait for a second
}
\end{lstlisting}}
\end{frame}


\section{Demonstration}
\subsection{``Hello Word!''}
\begin{frame}[containsverbatim]
\frametitle{Demonstration : ``Hello Word!''}
\begin{lstlisting}
require("arduino_wrapper")
app = App:new("Hello Word!")
app:run()
\end{lstlisting}
\end{frame}

\subsection{``Blink with button''}
\begin{frame}[containsverbatim]
\frametitle{Demonstration : ``Blink with button'' avec flash}
\tiny{\begin{lstlisting}
require("arduino_wraper")

function App:setup()
    self.ledpin = RED_LED
    pinMode(self.ledpin, OUTPUT)

    self.blink = false
    self:blink_toggle()
end

function App:loop()
    if self:btn_pressed() then
        self.blink = not self.blink
        self:blink_toggle()
    end
    delay(10)
end

function App:blink_toggle()
    [...]
end
\end{lstlisting}}
\end{frame}

\subsection{``Ascii table''}
\begin{frame}[containsverbatim]
\frametitle{Demonstration : Lancement de script sans flash}
\tiny{\begin{lstlisting}
require("arduino_wrapper")

function App:setup()
    self.byte = 33
end

function App:loop()
    self:println()
    self:write(self.byte)
    self:print(", dec: " .. self.byte)
    self:print(", hex: "), self:print(self.byte, HEX)
    self:print(", oct: "), self:print(self.byte, OCT)
    self:print(", bin: "), self:print(self.byte, BIN)

    self.byte = self.byte + 1
    delay(1000)
end

app = App:new("ASCII Table ~ Character Map")
app:run()

\end{lstlisting}}
\end{frame}

\section{Conclusion}
\begin{frame}
\frametitle{Conclusion}
Couplé à la puissance d'eLua, Easy-eLua permet :
\begin{itemize}
\item Débuter dans la programmation pour l'embarqué.
\item Portabilité : Le code Lua produit est compatible avec différentes architectures supportant elua.
\item Le RAD pour l'embarqué: Prototyper et expérimenter des applications rapidement. Testez vos idées directement sans besoin de simulations ou de futures modifications.
\item Flexibilité : Lua, langage de programmation de haut niveau, permet toute sorte d'utilisation.
\end{itemize}
\end{frame}


\begin{frame}
\frametitle{Remerciements}

\textbf{James Snyder:} ancien étudiant de l'Université de Northwestern en ingénerie en biotechnologie. Actuellement, il travaille au laboratoire Neuromech 
à Northwestern. Il est un utilisateur et collaborateur experimenté de eLua. Il a été un des pioniers pour la portabilité du eLua
dans la carte STM32F4-Discovery.
\end{frame}

\begin{frame}
\frametitle{Conclusion}
\begin{center}
\huge{Des questions ?}
\end{center}
\end{frame}

\end{document}

