%%%%%%%%%%%%%%%%%%%%%%%%%%%%%%%%%%%%%%%%%%%%%%%%%%%%%%%%%
\part{Travail réalisé}
\newpage
Avant de nous attarder sur le travail réalisé, nous allons tout d’abord aborder les méthodes de travail et notre organisation tout au long pour mener à bien ce projet. Puis nous détaillerons la structure du projet et les différentes composantes de celui-ci.  Nous détaillerons enfin les fonctions portées, les fonctions qu’ils restent à écrire et les concepts nouveaux, propres à Easy-eLua. Nous terminerons  par des exemples de programme développés avec Easy-eLua
%\newpage

\chapter[Organisation du travail]{Organisation du travail}
\label{chap:chap5}

Sur ce projet, il y a eu deux phases distinctes. La première phase concernait la recherche d’information et l’évaluation de la 
faisabilité du projet. Cette phase était assez longue et peu productive en code.  C’est également dans cette phase où l’on a commencé 
l’initiation à la programmation de l’embarqué et l’évaluation des différentes librairies disponibles pour linux (stlink, libopenstm32...).
 Nous nous concertions avec le tuteur pour jauger la voie à emprunter, car il faut bien le dire, beaucoup de sprint n’aboutissait 
pas forcément à un résultat espéré.

Avec le recul, on peut voir cette phase comme une longue formation aux différents outils et à la programmation de l’embarqué.
Elle est d’autant plus importante, qu’elle a permis à cette d’avoir une deuxième phase très rapide. 

Pour la deuxième phase, nous avons utilisé une méthode de travail qui s'inspire des méthodes agiles. Cette méthode s'appuie sur des 
valeurs fondamentales :

\begin{itemize}
 \item Les interactions entre individus priment sur les processus et outils, ceci permet de développer davantage le travail en groupe, et favorise la communication en face à face. 
 \item Le fonctionnement prime sur le reste: il est vital que l'application fonctionne. Le reste, et notamment la documentation technique, 
est une aide précieuse mais non un but en soi.
 \item Accepter le changement plutôt que de suivre un plan. En effet, il faut considérer le changement comme une opportunité car effectuer un changement au plus tôt  permet de réduire le coût. 
\end{itemize}
 
Ceci se traduit concrètement, par des objectifs courts, les ``sprints'' qu'on se fixe en fin de semaine pour la semaine à venir. Chaque fin de journée, on se regroupe durant 15-20 minutes pour une réunion hebdomadaire. À tour de rôle, chaque développeur prend la parole pour faire part de l’avancement de son projet et peut aborder les problèmes rencontrés.
Cela nous permet de connaître l’état d’avancement de l’autre développeur et de proposer des suggestions et un nouveau point de vue. Ça permet
aussi de voir si quelqu'un est bloqué, mais généralement si c'est le cas, l'aide survient plus tôt.

À la fin de chaque sprint, on effectue une réunion où l'on dresse le  bilan de ce qui a été réalisé puis l'on fixe les objectifs du sprint suivant. 
Il n’y a pas réellement de scrum master, vu qu’on est deux et avec le même niveau de connaissances. On décide tous les deux des sprints de chacun.